\documentclass[a4paper, czech]{article}

\title{Úloha č.4: Bipolární tranzistor jako zesilovač}
\author{Karolína Andrea Šebestová}
\date{Datum měření: 28.3.2024}

\usepackage[czech]{babel}
\usepackage{indentfirst}
\usepackage{graphicx}
\usepackage{float}
\usepackage[margin=1.5cm]{geometry}
\usepackage{booktabs}
\usepackage{amsmath}
\usepackage{multirow}
\usepackage{colortbl}

\begin{document}

\maketitle

\section{Teoretický úvod}

Tranzistor se nejčastěji používá pro zesilování signálů. Pro jeho správnou funkci je třeba nastavit jeho pracovní bod. Ten je určen stejnosměrnými napětími a proudy na elektrodách tranzistoru, která jsou taková, aby emitorový přechod byl v propustném směru a kolektorový v závěrném a aby se tranzistor nacházel při nulovém střídavém vstupním napětí ve vhodné části svých charakteristik. Zapojení zesilovače se společným emitorem se základním nastavením pracovního bodu tranzistoru pomocí předřadného rezistoru $R_1$ je na Obr. 1. Je nutné poznamenat, že v praxi se používají složitější obvody pro nastavení pracovního bodu zejména kvůli zajištění jeho teplotní stability.

\begin{figure}[H]
    \centering
    \includegraphics{zapojeni_SE.png}
    \caption{Zapojení se společným emitorem}
\end{figure}

\section{Seznam přístrojů}

\begin{enumerate}
    \item Osciloskop Agilent DSO-X 2002A
    \item Osciloskopická sonda HP 10074B
    \item Generátor Agilent 33521A
    \item Zdroj Agilent E3620A
    \item Multimetr Keysight 34461A
\end{enumerate}

\section{Úkoly měření}

Úkolem je změřit amplitudové kmitočtové a převodní charakteristiky bipolárního tranzistoru v zapojení se společným emitorem.

\begin{enumerate}
    \item Sestavte obvod podle Obr. 1. Přiveďte napájecí napětí $U_{CC} = 10 V$ a voltmetrem zkontrolujte, že napětí mezi kolektorem a emitorem $U_{CE}$ bez připojeného generátoru G je přibližně polovina napájecího napětí, tedy cca 5 V. Oba kanály osciloskopu nastavte na střídavou vazbu (stiskněte tlačítko s číslem kanálu 1 nebo 2, poté tlačítkem pod displejem Coupling vyberete AC). Signál do druhého kanálu osciloskopu přivádějte pomocí sondy. Černé krokodýlky koaxiálních kabelů a sondy připojujte k zemnímu, tedy ve schématu spodnímu, vodiči. Aby osciloskop ukazoval na kanále se sondou správné napěťové hodnoty, zmáčkněte tlačítko čísla kanálu 2, poté pod displejem Probe a v položce Probe nastavte pomocí kolečka s prosvětlenu šipkou 10.0:1. Vzhledem k malým hodnotám vstupního napětí je vhodné osciloskop synchronizovat na výstupní napětí, tedy na kanál 2. To provedete stiskem tlačítka Trigger a na displeji zvolíte Source 2. Knoflíkem Trigger pak nastavte úroveň synchronizace zhruba doprostřed průběhu na kanálu 2.   
    \item Změřte závislost zesílení zesilovače $A_{UdB} = 20 log ( U_{vyst} / U_{vst} )$ na frekvenci. Vstupní napětí nastavte přibližně 40 mV špička-špička (na generátoru nutno nastavit polovinu, tedy 20 mVpp), sinusový průběh a frekvenci měňte od 100 Hz do 5 MHz v přibližně 20 hodnotách. Pro aktivaci výstupu generátoru je třeba stisknout tlačítko Channel a poté tlačítkem pod displejem nastavit Output on. Frekvenční krok volte přibližně logaritmicky a zjemněte jej při větších změnách výstupního napětí, aby bylo možné v grafu dobře spojit měřené body hladkou čarou. Vstupní i výstupní napětí ($U_{vst}, U_{vyst}$) měřte na osciloskopu pomocí funkce tlačítka Meas, tlačítkem Type pod displejem zvolte Peak-Peak a tlačítkem Source vyberte měřený kanál. Volbu potvrďte pomocí Add Measurement.
    \item Změřenou závislost zesílení na frekvenci vyneste do grafu. Na vodorovné (frekvenční) ose použijte logaritmické měřítko, na svislé ose (zesílení $A_{UdB}$ v dB) použijte měřítko lineární. Odečtěte maximální hodnotu zesílení a kmitočet, na kterém toto maximum nastává. Zjistěte mezní kmitočet, kde klesne zesílení o 3 dB oproti maximálnímu zesílení.
    \item Změřte převodní charakteristiku zesilovače, tedy závislost velikosti výstupního napětí Uvýst na vstupním napětí $U_{vst}$. Použijte opět zapojení podle Obr. 1, na generátoru nastavte sinusový průběh, kmitočet 1 kHz a vstupní napětí $U_{vst}$ měňte od 10 mV do 100 mV špička-špička (na generátoru 5 mVpp až 50 mVpp). Změřte přibližně 20 hodnot. Na osciloskopu odečítejte mezivrcholové hodnoty vstupního a výstupního napětí opět pomocí funkce Meas. Časové průběhy vstupního a výstupního napětí pro $U_{vst}$ 10 mV a 100 mV špička-špička vložte (jako čitelnou fotku obrazovky nebo uložením na USB) či obkreslete do protokolu. Graficky znázorněte závislost výstupního napětí na vstupním s lineárními měřítky na obou osách. Zhodnoťte, ve které oblasti lze charakteristiku považovat za lineární. Také popište na osciloskopu pozorované změny v časovém průběhu výstupního napětí při zvyšování vstupního napětí.
\end{enumerate}

\section{Naměřené a vypočtené hodnoty}

V této sekci se nachází dvě tabulky.
Naměřené i vypočtené hodnoty amplitudové kmitočtové charakteristiky jsme vynesli do jedné společné tabulky.
Ve druhé tabulce se nachází naměřené hodnoty přenosové charakteristiky.

Dále se zde nachází i časové průběhy vstupního a výstupního napětí pro $U_{vst}$ 10 mV a 100 mV špička-špička uložené z osciloskopu.

Slovní popis symbolů veličin:
$f$ - frekvence;
$U_{vst}$ - vstupní napětí špička-špička;
$U_{vyst}$ - výstupní napětí špička-špička;
$A_{UdB}$ - zesílení zesilovače v dB

\begin{table}[H]
    \centering
    \begin{tabular}{ccccc}
        \multicolumn{5}{c}{$U_{CC} = 10V$} \\
        \toprule
        $*$  & $f\ (Hz)$    & $U_{vst}\ (mVpp)$ & $U_{vyst}\ (Vpp)$ & $A_{UdB}\ (dB)$ \\
        \midrule
        1  & 100       & 41        & 3,9       & 39,57     \\
        2  & 250       & 41        & 4,7       & 41,19     \\
        3  & 500       & 41        & 5         & 41,72     \\
        4  & 750       & 41        & 5,1       & 41,90     \\
        5  & 1 000     & 41        & 5,1       & 41,90     \\
        6  & 2 500     & 41        & 5,1       & 41,90     \\
        7  & 5 000     & 41        & 5,1       & 41,90     \\
        8  & 7 500     & 41        & 5,1       & 41,90     \\
        9  & 10 000    & 41        & 5,1       & 41,90     \\
        10 & 25 000    & 41        & 5,1       & 41,90     \\
        11 & 50 000    & 41        & 4,9       & 41,55     \\
        12 & 75 000    & 41        & 4,8       & 41,37     \\
        13 & 100 000   & 41        & 4,7       & 41,19     \\
        14 & 250 000   & 41        & 4,5       & 40,81     \\
        15 & 500 000   & 40        & 4,3       & 40,63     \\
        16 & 750 000   & 39        & 4,1       & 40,43     \\
        17 & 1 000 000 & 38        & 3,9       & 40,23     \\
        18 & 2 500 000 & 33        & 2,5       & 37,59     \\
        19 & 5 000 000 & 33        & 1,2       & 31,21     \\
        \bottomrule
    \end{tabular}
    \caption{Naměŕené a vypočtené hodnoty amplitudové kmitočtové charakteristiky}
\end{table}

\begin{table}[H]
    \centering
    \begin{tabular}{ccc}
        \multicolumn{3}{c}{$f = 1kHz$} \\
        \toprule
        $*$  & $U_{vst}\ (mVpp)$ & $U_{vyst}\ (Vpp)$ \\
        \midrule
        1  & 11        & 1,4       \\
        2  & 16        & 2,1       \\
        3  & 21        & 2,7       \\
        4  & 27        & 3,3       \\
        5  & 31        & 3,9       \\
        6  & 36        & 4,4       \\
        7  & 41        & 5,1       \\
        8  & 46        & 5,5       \\
        9  & 51        & 6,1       \\
        10 & 55        & 6,6       \\
        11 & 60        & 7         \\
        12 & 65        & 7,5       \\
        13 & 70        & 7,9       \\
        14 & 76        & 8,4       \\
        15 & 80        & 8,7       \\
        16 & 85        & 9         \\
        17 & 90        & 9,1       \\
        18 & 95        & 9,2       \\
        19 & 100       & 9,3       \\
        \bottomrule
    \end{tabular}
    \caption{Naměřené hodnoty přenosové charakteristiky}
\end{table}

Na následujícím naměřeném průběhu je patrné, že při nízkém vstupním napětí $U_{vst}$ je patrné, že nedochází ke zkreslení výstupného signálu.

\begin{figure}[H]
    \centering
    \includegraphics{prubeh_10mVpp.png}
    \caption{Časový průběh vstupního a výstupního napětí pro $U_{vst} =  10 mV$}
\end{figure}

Na průběhu s vyšším vstupním napětím $U_{vst}$ je však zřejmé, že dochází ke značnému zkreslení. Je patrné, že dochází k "ořezu" ve vrchní části průběhu.

\begin{figure}[H]
    \centering
    \includegraphics{prubeh_100mVpp.png}
    \caption{Časový průběh vstupního a výstupního napětí pro $U_{vst} =  100 mV$}
\end{figure}

\section{Příklady výpočtu}

Závislost zesílení zesilovače $A_{UdB}$ na frekvenci $f$ byla vypočtena podle následujícího vztahu:

$$A_{UdB}=20 \cdot \log\left(\frac{U_{vyst}}{U_{vst}}\right)$$

Uvedeme si konkrétní příklad výpočtu pro první řádek tabulky naměřených hodnot:

$${A_{UdB}}_1 = 20 \cdot \log\left(\frac{U_{vyst}}{U_{vst}}\right) = 20 \cdot \log\left(\frac{3,9 V}{41 \cdot 10^{-3}V}\right) \approx \underline{\underline{39,57 dB}}$$

\section{Grafy}

Naměřené hodnoty zesílení $A_{UdB}$ byly vyneseny do grafu jako funkce frekcence $f$, která je zobrazena v logaritmickém měřítku. Výsledný graf tedy tvoří amplitudovou kmitočtovou charakteristiku.
Maximální hondota zesílení je $A_{UdB} = 41,90dB$ a toto maximum nastává na kmitočtovém intervalu $750 Hz$ až $25 kHz$.
Mezní kmitočet, kde zesílení klesne o $3 dB$ je roven hodnotě $38,9 dB$ oproti maximální mu zesílení byl odečten z grafu a přibližně se rovná hodnotě $1,6 MHz$.

\begin{figure}[H]
    \centering
    \includegraphics{akch.png}
    \caption{Naměřená amplitudová kmitočtová charakteristika vynesena do grafu}
\end{figure}

Převodní charakteristika zesilovače je zobrazena jako závislost naměřených mezivrcholových hodnot výstupního napětí $U_{vyst}$ na vstupním napětí $U_{vst}$.
Ve vyhotoveném grafu je oblast, kterou je možné považovat za lineární ilustračně proložena lineární funkcí.
V posledních čtyřech naměŕených bodech dochází k velkému "ohybu" charakteristiky, proto nejsou zahrnuty do lineární části.

\begin{figure}[H]
    \centering
    \includegraphics{pch.png}
    \caption{Naměřená přenosová charakteristika vynesena do grafu}
\end{figure}

\section{Závěr}

\end{document}