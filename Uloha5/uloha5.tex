\documentclass[a4paper, czech]{article}

\title{Úloha č.5: Tranzistor MOSFET - V-A charakteristiky a použití jako spínač}
\author{Karolína Andrea Šebestová}
\date{Datum měření: 11.4.2024}

\usepackage[czech]{babel}
\usepackage{indentfirst}
\usepackage{graphicx}
\usepackage{float}
\usepackage[margin=1.5cm]{geometry}
\usepackage{booktabs}
\usepackage{amsmath}
\usepackage{multirow}
\usepackage{colortbl}
\usepackage{array}

\begin{document}

\maketitle

\section{Teoretický úvod}

V této úloze budou měřeny vlastnosti unipolárního tranzistoru MOSFET s indukovaným kanálem typu N (NMOS), jehož princip činnosti je naznačen na Obr. 1.
Základem tranzistoru je polovodičová destička (substrát) typu P, ve které jsou vytvořeny dvě oblasti s vodivostí typu N.
K nim jsou připojeny elektrody source (S) a drain (D).
Hradlo (G - gate) je od substrátu odděleno izolujícím oxidem křemíku ($SiO_2$).
Připojením kladného napětí $U_{GS}$ na hradlo dojde v substrátu k odpuzování majoritních nosičů náboje (tj. děr) a mezi D a S vzniká vodivý kanál (na Obr. 1 černě), kterým mohou procházet elektrony.
Čím větší bude napětí $U_{GS}$, tím bude kanál širší a tím bude větší proud $I_D$ (viz převodní charakteristika na Obr. 2).
Z výstupních charakteristik na Obr. 2 je zřejmé, že proud $I_D$ se zvětšuje také při zvětšování $U_{DS}$.
Pro $U_{DS} > U_{DS_{sat}}$ však přestane proud $I_D$ narůstat.
Při zvyšování $U_{DS}$ totiž zároveň klesá rozdíl potenciálů mezi G a D a v blízkosti D se kanál zužuje (Obr. 1).
Při nulovém rozdílu dojde na straně kolektoru (D) k uzavření kanálu a k saturaci proudu.

\begin{figure}[H]
    \centering
    \includegraphics{princip.png}
    \caption{Princip činnosti tranzistoru NMOS s indukovaným kanálem}
\end{figure}

\begin{figure}[H]
    \centering
    \includegraphics{charakteristiky.png}
    \caption{Charakteristiky tranzistoru NMOS s indukovaným kanálem}
\end{figure}

\section{Seznam přístrojů}

\begin{enumerate}
    \item Zdroj Agilent E3620A
    \item 2x Multimetr Keysight 34461A
    \item Generátor Agilent 33521A
    \item Osciloskop Agilent DSO-X 2002A
    \item Osciloskopická sonda HP 10074B
\end{enumerate}

\section{Úkoly měření}

\begin{enumerate}
    \item Změřte výstupní charakteristiky tranzistoru BS170, tedy $I_D = f (U_{DS})$ při různých hodnotách $U_{GS} = konst.$ v zapojení podle Obr. 3. Hodnoty napětí $U_{GS}$ volte 0; 2; 2,5; 2,8; 3 V a odečítejte je přímo na displeji laboratorního zdroje E3620A (levý voltmetr tedy nebude použit). Hodnoty $I_D$ a $U_{DS}$ měřte pomocí multimetrů. Napětí $U_{DS}$ nastavujte pomocí regulovatelného zdroje E3620A (druhý kanál) až do hodnoty $U_{CC} = 25 V$. Proměřte jednotlivé charakteristiky v dostatečném počtu bodů pro spojení hladkými křivkami, hodnoty zapište do tabulky a zakreslete do grafu.
    \item Změřte převodní charakteristiku $I_D = f (U_{GS})$ při $U_{DS} = konst.$ opět podle Obr. 3, tedy zapojení ponechte z předchozího úkolu. Nastavte $U_{DS} = 5 V$ (měřte jej multimetrem) a udržujte toto napětí pomocí regulace zdroje $U_{CC}$ během celého měření konstantní. Napětí $U_{GS}$ nastavujte podle displeje zdroje od nuly po maximum, kdy ještě půjde udržet $U_{DS} = 5 V$. Naměřené hodnoty zapište do tabulky a vyneste do grafu. Totéž měření proveďte ještě pro $U_{DS} = 3 V$. 
    \item Zapojte obvod pro spínání tranzistorem NMOS podle Obr. 4. Na generátoru nastavte obdélníkový průběh (Waveforms, Square), kmitočet 2 MHz, mezivrcholovou hodnotu napětí 5 V špička-špička (na generátoru nastavte 2,5 Vpp)  a ofset 2,5 V (na generátoru 1,25 V). Napětí $U_{CC}$ nastavte na 15 V. Signál do druhého kanálu osciloskopu přivádějte pomocí osciloskopické sondy (s přenosem 1:10, nezapomeňte ji nastavit na osciloskopu jako v minulých úlohách). Černé krokodýlky koaxiálních kabelů a sondy připojujte k zemnímu, tedy ve schématu spodnímu, vodiči. Z osciloskopu obkreslete či elektronicky vložte do protokolu vstupní a výstupní průběhy napětí a odečtěte časové posuny mezi vstupem a výstupem, jak je naznačeno v Obr. 5. Tyto posuny vyznačte v průbězích.
\end{enumerate}

\begin{figure}[H]
    \centering
    \includegraphics{zapojeni1.png}
    \caption{Zapojení pro měření charakteristik tranzistoru NMOS}
\end{figure}

\begin{figure}[H]
    \centering
    \includegraphics{zapojeni2.png}
    \caption{Zapojení tranzistoru NMOS jako spínač}
\end{figure}

\begin{figure}[H]
    \centering
    \includegraphics{prubehy.png}
    \caption{Průběhy napětí při spínání tranzistoru NMOS}
\end{figure}

\section{Naměřené hodnoty}

\subsection{Výstupní charakteristiky tranzistoru NMOS}

\begin{table}[H]
    \centering 
    \begin{tabular}{cccccccccc}
    \toprule
        \multicolumn{2}{c}{$U_{GS} = 0V$} & \multicolumn{2}{c}{$U_{GS} = 2V$} & \multicolumn{2}{c}{$U_{GS} = 2,5V$} & \multicolumn{2}{c}{$U_{GS} = 2,8V$} & \multicolumn{2}{c}{$U_{GS} = 3V$} \\ 
        $U_{DS}\ (V)$ & $I_D\ (\mu A)$ & $U_{DS}\ (V)$ & $I_D\ (\mu A)$ & $U_{DS}\ (V)$ & $I_D\ (mA)$ & $U_{DS}\ (V)$ & $I_D\ (mA)$ & $U_{DS}\ (V)$ & $I_D\ (mA)$ \\
        \midrule
        0  & 0     & 0    & 0      & 0     & 0     & 0    & 0     & 0     & 0     \\
        1  & 0,098 & 0,05 & 37,73  & 0,05  & 1,61  & 0,05 & 4,74  & 0,025 & 3,89  \\
        2  & 0,197 & 0,1  & 49,82  & 0,1   & 2,343 & 0,1  & 8,07  & 0,05  & 7,44  \\
        3  & 0,297 & 0,25 & 58,18  & 0,25  & 3,216 & 0,25 & 13,3  & 0,075 & 10,6  \\
        4  & 0,397 & 0,5  & 61,71  & 0,5   & 3,526 & 0,5  & 15,7  & 0,1   & 13,46 \\
        5  & 0,497 & 1    & 62,87  & 1     & 3,684 & 1    & 16,76 & 0,125 & 16,17 \\
        6  & 0,596 & 2    & 65,27  & 2     & 3,845 & 2    & 17,91 & 0,15  & 18,51 \\
        7  & 0,695 & 3    & 67,2   & 3     & 3,99  & 3    & 18,79 & 0,175 & 20,4  \\
        8  & 0,795 & 4    & 69,08  & 4     & 4,11  & 3,5  & 19,66 & 0,2   & 22,17 \\
        9  & 0,894 & 5    & 71,03  & 5     & 4,23  & 3,77 & 20,88 & 0,21  & 22,85 \\
        10 & 0,994 & 6    & 72,99  & 6     & 4,4   &      &       & 0,22  & 23,46 \\
        11 & 1,093 & 7    & 75,11  & 7     & 4,55  &      &       & 0,23  & 24,13 \\
        12 & 1,192 & 8    & 77,35  & 8     & 4,71  &      &       & 0,238 & 24,51 \\
        13 & 1,292 & 9    & 79,71  & 9     & 4,91  &      &       &       &       \\
        14 & 1,392 & 10   & 82,29  & 10    & 5,1   &      &       &       &       \\
        15 & 1,493 & 11   & 84,97  & 11    & 5,31  &      &       &       &       \\
        16 & 1,592 & 12   & 87,79  & 12    & 5,53  &      &       &       &       \\
        17 & 1,691 & 13   & 90,78  & 13    & 5,86  &      &       &       &       \\
        18 & 1,791 & 14   & 93,92  & 14    & 6,14  &      &       &       &       \\
        19 & 1,89  & 15   & 97,44  & 15    & 6,43  &      &       &       &       \\
        20 & 1,989 & 16   & 101,25 & 16    & 6,76  &      &       &       &       \\
        21 & 2,09  & 17   & 105,22 & 17    & 7,17  &      &       &       &       \\
        22 & 2,188 & 18   & 109,38 & 17,38 & 7,5   &      &       &       &       \\
        23 & 2,287 & 19   & 113,96 &       &       &      &       &       &       \\
        24 & 2,387 & 20   & 118,66 &       &       &      &       &       &       \\
        25 & 2,486 & 21   & 123,91 &       &       &      &       &       &       \\
           &       & 22   & 129,51 &       &       &      &       &       &       \\
           &       & 23   & 135,42 &       &       &      &       &       &       \\
           &       & 24   & 141,73 &       &       &      &       &       &       \\
           &       & 25   & 148,74 &       &       &      &       &       &       \\
    \bottomrule
    \end{tabular}
    \caption{sdfasd}
\end{table}

\subsection{Převodní charakteristiky tranzistoru NMOS}

\begin{table}[H]
    \centering
    \begin{tabular}{ccc}
        \toprule
        \multirow{2}{*}{$U_{GS}\ (V)$}& $U_{DS}=5V$ & $U_{DS}=3V$ \\
         & $I_D\ (mA)$ & $I_D\ (mA)$ \\
        \midrule
        0   & 0,00  & 0,00  \\
        1   & 0,00  & 0,00  \\
        2   & 0,08  & 0,07  \\
        2,1 & 0,21  & 0,18  \\
        2,2 & 0,48  & 0,47  \\
        2,3 & 1,09  & 1,00  \\
        2,4 & 2,21  & 2,10  \\
        2,5 & 4,30  & 3,93  \\
        2,6 & 7,63  & 7,03  \\
        2,7 & 12,73 & 11,82 \\
        2,8 & 20,12 & 18,49 \\
        \bottomrule
    \end{tabular}
    \caption{aldeskgfajkd}
\end{table}

\subsection{Časové posuny při zapojení tranzistoru NMOS jako spínač}

\begin{figure}[H]
    \centering
    \includegraphics[width=\textwidth]{t_df.png}
    \caption{Průběhy napětí při spínání tranzistoru NMOS}
\end{figure}

\begin{figure}[H]
    \centering
    \includegraphics[width=\textwidth]{t_f.png}
    \caption{Průběhy napětí při spínání tranzistoru NMOS}
\end{figure}

\begin{figure}[H]
    \centering
    \includegraphics[width=\textwidth]{t_dr.png}
    \caption{Průběhy napětí při spínání tranzistoru NMOS}
\end{figure}

\begin{figure}[H]
    \centering
    \includegraphics[width=\textwidth]{t_r.png}
    \caption{Průběhy napětí při spínání tranzistoru NMOS}
\end{figure}

\begin{table}[H]
    \centering
    \begin{tabular}{cccc}
        \toprule
        $t_{df}\ (ns)$ & $t_{f}\ (ns)$ & $t_{dr}\ (ns)$ & $t_{dr}\ (ns)$ \\
        \midrule
        9,5 & 14,5 & 21 & 74 \\
        \bottomrule
    \end{tabular}
    \caption{aldeskgfajkd}
\end{table}

\section{Grafy}

\begin{figure}[H]
    \centering
    \includegraphics[width=\textwidth]{vystupni.png}
    \caption{Průběhy napětí při spínání tranzistoru NMOS}
\end{figure}

\begin{figure}[H]
    \centering
    \includegraphics{prevodni.png}
    \caption{Průběhy napětí při spínání tranzistoru NMOS}
\end{figure}

\section{Závěr}

\end{document}